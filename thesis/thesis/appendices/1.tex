\documentclass[class=article, crop=false]{standalone}
\usepackage[utf8]{inputenc}


\usepackage{amsfonts}
\usepackage{amsmath}
\usepackage[english]{babel}
\usepackage{booktabs}
\usepackage{caption}
\usepackage{graphicx}
\usepackage{import}
\usepackage{multicol}
\usepackage{multirow}
\usepackage[subpreambles=false]{standalone}
\usepackage{subcaption}
\usepackage{tikz}

\usepackage[linesnumbered,ruled,vlined]{algorithm2e}
\newcommand\commentfont[1]{\footnotesize\ttfamily\textcolor{blue}{#1}}
\SetCommentSty{commentfont}

\usepackage[backend=biber]{biblatex}
\addbibresource{references.bib}

\usepackage{geometry}
\geometry{
   a4paper,
   left=20mm, right=20mm,
   top=20mm, bottom=25mm
}
 
\usepackage{hyperref}
\hypersetup{
  colorlinks=true,
  linkcolor=blue,
  citecolor=black,
}

\usepackage{pgfplots}
\pgfplotsset{compat=1.3}
 
 
\setlength{\tabcolsep}{2pt} % Default value: 6pt
\renewcommand{\arraystretch}{1} % Default value: 1


\begin{document}




\section{Appendix title}
\label{sec:app1}
explain different transformations (rigid...), centering, rotating..

\section{Background for choice of the network}
\label{sec:app2}
The project started with exploring different networks that can tackle the problem of 3D landmark detection. This phase also lead to insights regarding networks that do not work well for the problem at hand. PointNet is one of the earliest and simpler model architectures that operates on point clouds was a straightforward choice  We tried the Pytorch implementation of the extension of PointNet, called PointNet++. The extensio in MeshCNN and Pointnet; many network architectures don't scale well


\end{document}
