\documentclass[crop=false]{standalone}

\usepackage[subpreambles=false]{standalone}
\usepackage{import}
\usepackage{graphicx}
\usepackage{subcaption}
\usepackage{tikz}

\begin{document}

\vspace{-0.015\linewidth}
\begin{center}
  % \hspace*{-0.08\linewidth}
  \includegraphics[width=1\linewidth]{thesis/appendices/import/imgs/hks_dragon.png}
  %\vspace*{-0.06\linewidth}
  \captionof{figure}{
    \textbf{Heat Kernel Signature.}
    \small At four different points on a triangulated surface. For small values of t ($t < 1$), the heat kernel signature $k_t(x,x)$ is almost the same. This is, because the local geometry, i.e. the tips of the dragon's feet is approximately equivalent. As t increases ($t > 1$), more global information is considered and the heat kernel signatures diverge, whereas the two points at the front feet (1, 2) and the two points at the back feet (3, 4) still capture more common surface information about the shape compared to one at the front and one at the back. Image from \cite{Sun2009}.
  }
  \label{fig:rot_trans}
\end{center}
\vspace{-0.01\linewidth}

\end{document}