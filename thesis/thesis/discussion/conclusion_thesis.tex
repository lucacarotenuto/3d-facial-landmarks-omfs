\documentclass[class=article, crop=false]{standalone}

\usepackage[backend=biber]{biblatex}
\addbibresource{references.bib}

\usepackage{hyperref}

 \usepackage{geometry}
\geometry{
   a4paper,
   left=20mm, right=20mm,
   top=20mm, bottom=25mm
 }
 
\usepackage{multicol}
\usepackage[subpreambles=false]{standalone}

\begin{document}




\section{Conclusion}
\label{sec:conclusion}

% Further work: focal loss?

%%%%%%%%%%%%%%%%%%%%%%%%%%%%%%%% summary
In this work, a facial landmark detection network is developed based on DiffusionNet. An initial network predicts rough landmark positions in low resolution, which can be tweaked by a refinement model that operates in full resolution. With a mean error of 2.22mm, the landmark detector shows promising results and makes only slightly more inaccurate predictions than a human annotator does. However, the model is limited to consistent head orientations. If that requirement is not met, a model trained on HKS features can enable rotation invariance but shows inferior detection accuracies.

%%%%%%%%%%%%%%%%%%%%%%%%%%%%%%%% implications

It is recommended to further research methods to promote rigid invariance as it has high clinical relevance. 3D deep learning is a challenging field that only recently is receiving increasing attention from the research community. Current methods have promising potential and better prevent overfitting on the mesh sampling and data representation.
% neural networks quickly reach memory limits

%%%%%%%%%%%%%%%%%%%%%%%%%%%%%%%% conclusion


\end{document}
